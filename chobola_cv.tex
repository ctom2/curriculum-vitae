\documentclass[10pt]{article}

\usepackage{hyperref}
\hypersetup{
    colorlinks=true,  
    urlcolor=blue,
}

\usepackage{helvet} % Default font is the helvetica postscript font
%\usepackage{newcent} % To change the default font to the new century schoolbook postscript font uncomment this line and comment the one above

\usepackage[a4paper, total={7in, 10.5in}]{geometry}

\usepackage[dvipsnames]{xcolor}
\usepackage[super]{nth}
\setlength{\parindent}{0em}
\pagenumbering{gobble}

\usepackage{titlesec}
\usepackage{lipsum}

\begin{document}


% ======================================================================
\begin{center}
    \begin{huge}
        Tomáš Chobola\\
    \end{huge}
\end{center}
\vspace{-1em}
\begin{center}
    \begin{small}
        M.Sc. Student $\cdot$ Data Engineering and Analytics $\cdot$ Munich, DE\\
    \end{small}
\end{center}
\vspace{-1em}
\begin{center}
    \begin{small}
        +420 604 967 459
        $\cdot$ 
        \href{mailto:tchobola@protonmail.com}{tchobola@protonmail.com}, \href{mailto:chobola@in.tum.de}{chobola@in.tum.de}
        $\cdot$
        \href{https://github.com/ctom2}{github.com/ctom2}\\
    \end{small}
\end{center}

% ======================================================================

\begin{Large}
    \textsc{Education}
    \vspace{0.4em}
    \hrule
    \vspace{0.4em}
\end{Large}

%%%
\begin{large}
    Technical University of Munich\\[0.2em]
\end{large}
\begin{normalsize}
    \textbf{M.Sc. Data Engineering and Analytics.}\hfill November 2020--\\[0.8em]
\end{normalsize}
%%%
\begin{large}
    Czech Technical University in Prague\\[0.2em]
\end{large}
\begin{normalsize}
    \textbf{Bc. Knowledge Engineering.}\textit{ CERN certified thesis}:\hfill October 2016--July 2020\\[0.2em]
\end{normalsize}
\begin{normalsize}
    \href{http://cds.cern.ch/record/2722599?ln=en}{Study of light-by-light scattering with the ATLAS Forward Proton (AFP) Detector at CERN}\\[0.8em]
\end{normalsize}
%%%
\begin{large}
    The Hong Kong Polytechnic University\\[0.2em]
\end{large}
\begin{normalsize}
    Computer Science, Study Abroad.\hfill August 2019--January 2020\\
\end{normalsize}

% ======================================================================

\begin{Large}
    \textsc{Experience}
    \vspace{0.4em}
    \hrule
    \vspace{0.4em}
\end{Large}

\begin{large}
    eClub Prague\hfill
    \begin{small}
        February 2020--December 2020\\[0.2em]
    \end{small}
\end{large}
\begin{normalsize}
    \textit{Conversational AI Researcher}. Remote work, 1-2 days/week.\\[0.2em]
    Using TensorFlow and PyTorch to adapt Transformers and other NLP models for the end-of-utternace detection.\\
\end{normalsize}

\begin{large}
    Faculty of Information Technology, CTU\hfill
    \begin{small}
        July 2020--November 2020\\[0.2em]
    \end{small}
\end{large}
\begin{normalsize}
    \textit{Research Assistant}.\\[0.2em]
    Researching and extending the state-of-the-art models for few-shot classification, and implementing them in PyTorch and Tensorflow.\\
\end{normalsize}

\begin{large}
    Faculty of Information Technology, CTU\hfill
    \begin{small}
        May 2019--August 2019\\[0.2em]
    \end{small}
\end{large}
\begin{normalsize}
    \textit{Research Assistant}.\\[0.2em]
    Researching symmetric heterogeneous transfer-learning models with auto-encoders and GANs for image-to-image translation setting, and implementing them in Tensorflow.\\
\end{normalsize}

\begin{large}
    Institute of Experimental and Applied Physics, CTU\hfill
    \begin{small}
        February 2019--July 2020\\[0.2em]
    \end{small}
\end{large}
\begin{normalsize}
    \textit{Research Assistant}.\\[0.2em]
    Applying regression tools (Python, C++), mainly Gaussian Process, and Monte Carlo simulations in analysis of data obtained from the ATLAS experiment. Working closely with the international collaboration of researchers at CERN.\\
\end{normalsize}

\begin{large}
    Commerzbank AG, Prague\hfill
    \begin{small}
        March 2019--July 2019\\[0.2em]
    \end{small}
\end{large}
\begin{normalsize}
    \textit{Junior Software Specialist}. 2-3 days/week.\\[0.2em]
    Carrying out workflow and database improvements with bash and SQL scripting.\\
\end{normalsize}

% ======================================================================

\begin{Large}
    \textsc{Publications}
    \vspace{0.4em}
    \hrule
    \vspace{0.4em}
\end{Large}
% \vspace{5pt}
\begin{normalsize}
    \textit{Transfer learning based few-shot classification using optimal transport mapping from preprocessed latent space of backbone neural network}. \href{https://arxiv.org/abs/2102.05176}{arXiv:2102.05176}.\\
    \textbf{T. Chobola}, D. Vašata and P. Kordík. \textit{Meta-Learning Workshop, \nth{35} AAAI Conference on Artificial Intelligence}. Virtual. February 2021.\\
    
   \textit{Unsupervised Latent Space Translation Network}. \href{https://arxiv.org/abs/2003.09149}{arXiv:2003.09149}.\\
   M. Friedjungová, D. Vašata, \textbf{T. Chobola}, M. Jiřina. \textit{In Proceedings of \nth{28} European Symposium on Artificial Neural Networks, Computational Intelligence and Machine Learning}. Bruges, Belgium. October 2021.\\
\end{normalsize}

% ======================================================================
\begin{Large}
    \textsc{Talks \& Presentations}
    \vspace{0.4em}
    \hrule
    \vspace{0.4em}
\end{Large}

\begin{normalsize}
    \textbf{DPG-Frühjahrstagungen (DPG Spring Meetings)}.\\
    Virtual conference, presenting data analysis of the ATLAS experiment at CERN. March 15, 2021.\\
\end{normalsize}

\begin{normalsize}
    \textbf{AAAI Workshop on Meta-learning with co-hosted Competition}.\\
    Virtual conference, contributed talk and poster presentation. February 9, 2021.\\
\end{normalsize}
\newpage
% ======================================================================

\begin{Large}
    \textsc{Grants \& Awards}
    \vspace{0.4em}
    \hrule
    \vspace{0.4em}
\end{Large}
\begin{normalsize}
    \textbf{AAAI 2021 MetaDL Challenge \nth{2} Place Prize}.\\[0.2em]
    \textbf{Student Summer Research Program 2020 of FIT CTU in Prague Grant}. Support of the few-shot meta-learning research project.\\[0.2em]
    \textbf{Student Summer Research Program 2019 of FIT CTU in Prague Grant}. Support of the image transfer-learning research project.
\end{normalsize}
\vspace{20pt}

% ======================================================================

\begin{Large}
    \textsc{Extracurricular activities}
    \vspace{0.4em}
    \hrule
    \vspace{0.4em}
\end{Large}

\begin{normalsize}
    CTU Toastmasters club member. 2019.\\[0.2em]
    Organising the IAESTE Job Fair at CTU in Prague. 2018.\\[0.2em]
    Organising the international symposium Humans in Space and assisting the scientific committee. 2015.\\
\end{normalsize}

% ======================================================================

\begin{Large}
    \textsc{Skills}
    \vspace{0.4em}
    \hrule
    \vspace{0.4em}
\end{Large}
% \vspace{5pt}
\begin{normalsize}
    \textbf{Programming languages}
    \begin{itemize}\itemsep -2pt % Reduce space between items
        \item Working knowledge of C/C++
        \item Python for scripting, data analysis and machine learning applications and modelling
        \item Basic knowledge of Java
        \item Data analysis and visualisation with R
        \item Query optimisation and database administration with PostreSQL
        \item Version control systems (Git)
        \item Documentation with LaTeX and Markdown
        \item Unix-based systems knowledge
    \end{itemize}
    
    \textbf{Languages}
    \begin{itemize}\itemsep -2pt % Reduce space between items
        \item \textit{Czech} -- Native
        \item \textit{English} -- Professional working proficiency
        \item \textit{German} -- Elementary
    \end{itemize}
\end{normalsize}

% ======================================================================

\vspace{290pt}
\hfill May 10, 2021\hspace{1cm}Tomáš Chobola
\vspace{10pt}

I hereby authorise the use of my personal data in accordance to the GDPR 679/16 – “European regulation on the protection of personal data``.

\end{document}